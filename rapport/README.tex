\documentclass{article}

\usepackage[utf8]{inputenc}
\usepackage[T1]{fontenc}
\usepackage[francais]{babel}
\usepackage{amsmath}
\usepackage{amssymb}
\usepackage{mathrsfs}
\usepackage{amsthm}
\usepackage{float}
\usepackage{stmaryrd}
\usepackage{marvosym}
\usepackage{geometry}
\usepackage{setspace}
\usepackage{mathrsfs}

%\usepackage{fancyhdr}
%\usepackage{lastpage}
% \usepackage{nicefrac}
\usepackage{mathpartir}

\title{\titre}
\date{}
\author{\auteur}



\makeatletter
\newlength{\boxed@align@width}
\newcommand{\boxedalign}[2]{
	#1 & \setlength{\boxed@align@width}{\widthof{$\displaystyle#1$}+0.1389em+\fboxsep+\fboxrule}
	\hspace{-\boxed@align@width}\addtolength{\boxed@align@width}{-\fboxsep-\fboxrule}\boxed{\vphantom{#1}\hspace{\boxed@align@width}#2}}
\makeatother

\newcommand{\titre}{Devoir Maison de Programmation Avancée : Mini-ML}
\newcommand{\titreCourt}{DM -- Mini-ML}
\newcommand{\auteur}{Marius \textsc{Dufraisse}}

\newcommand{\im}[1]{\Im m\left(#1\right)}
\newcommand{\re}[1]{\Re e\left(#1\right)}

\newcommand{\N}{\mathbbm{N}}
\newcommand{\R}{\mathbbm{R}}

\newcommand{\prop}{\mathcal{P}}

\newcommand{\sem}{\Rightarrow}

\newcommand{\parEnt}[1]{\lfloor #1\rfloor}

\newtheoremstyle{thm}{3pt}{3pt}{\itshape}{\parindent}{\bfseries}{.}{.5em}{}
\theoremstyle{thm}
\newtheorem*{proposition}{Proposition}
\newtheorem*{theoreme}{Théorème}
\newtheorem*{lemme}{Lemme}
\newtheorem*{conjecture}{Conjecture}
\newtheorem*{corollaire}{Corollaire}
\newtheorem*{contraposee}{Contraposée}
\newtheorem{corol}{Corollaire}
\newtheoremstyle{def}{3pt}{3pt}{}{\parindent}{\bfseries}{.}{.5em}{}
\theoremstyle{def}
\newtheorem*{definition}{Définition}
\newtheorem*{remarque}{Remarque}
\newtheorem*{remarques}{Remarques}
\newtheorem*{exemple}{Exemple}
\newtheoremstyle{dem}{3pt}{3pt}{}{\parindent}{\itshape}{.}{.5em}{}
\theoremstyle{dem}
\newtheorem*{dem}{Démonstration}

\newcommand{\diff}{\mathop{}\mathopen{}\mathrm{d}}
\newcommand{\trig}[1]{\left\langle #1 \right\rangle}
\newcommand{\trigb}[1]{\trig{#1}^\bullet}
\newcommand{\trigs}[1]{\trig{#1}^*}
\newcommand{\subs}[2]{\left[#1:=#2\right]}
\newcommand{\surj}[2]{\mathscr{S}\left(#1,#2\right)}
\newcommand{\context}[1]{\mathcal{C}\left[#1\right]}

\geometry{a4paper, top=2.0cm, bottom=2cm, left=2.5cm , right=2.5cm}

\newcommand{\defFun}[5]{
\begin{array}{ccccc}
	#1 & : & #2 & \to & #3 \\
	& & #4 & \mapsto & #5 \\
\end{array}}

\newcommand{\lambd}{$\lambda$}
\newcommand{\inter}[2]{\left\llbracket#1,#2\right\rrbracket}
\newcommand{\code}[1]{\texttt{#1}}
\newcommand{\codem}[1]{\mathtt{#1}}
\newcommand{\types}[1]{\mathbf{#1}}

\begin{document}
	\renewcommand{\labelitemi}{\textbullet}

	% \pagestyle{fancy}
	% \renewcommand{\headheight}{13.6pt}
	% \renewcommand{\headrulewidth}{0.4pt}
	% \renewcommand{\footrulewidth}{0.4pt}
	% \renewcommand{\headsep}{15pt}
	% \lhead{\auteur}
	% \chead{}
	% \rhead{\titreCourt}
	% \lfoot{\leftmark}
	% \cfoot{}
	% \rfoot{{\small Page \nicefrac{\thepage}{\pageref{LastPage}} }}


	\maketitle
	% \thispagestyle{fancy}
	\section{La spécification de Mini-ML.}

	\paragraph{Question 1.}
	Pour inclure les fonctions \code{pred} et \code{succ} on ajoute les règles de sémantique et de typage suivantes
	\[ \frac{t \sem u}{\codem{succ}t\sem\codem{succ}{u}}\qquad \frac{t \sem u}{\codem{pred}t\sem\codem{pred}{u}} \]
	\[\frac{\Gamma\vdash  n : \types{nat}}{\Gamma\vdash \codem{succ}n:\types{nat}} \qquad\frac{\Gamma\vdash  n : \types{nat}}{\Gamma\vdash \codem{pred}n:\types{nat}}\]

	\paragraph{Question 2.}
	On définit les fonctions \code{plus} et \code{fois} de la façon suivante
	\begin{align*}
		\codem{let\ rec}\ &plus\ = \lambda x^{\types{int}} .\\
		&\codem{let}\ aux = \\
		&\hspace{0.75cm}\lambda y^{\types{int}}.\codem{if\ }x=0\codem{then}\ y\ \codem{else}(plus(\codem{pred}x)(\codem{succ}y))\\
		\codem{in}\ &aux\\
		&\codem{in\ }plus
	\end{align*}

	\begin{align*}
		\codem{let\ rec}\ &fois=\lambda x^{\types{int}} .\\
		&\codem{let}\ aux = \\
		&\hspace{0.75cm}\lambda y^{\types{int}}.\codem{if\ }x=0\codem{then}\ 0\ \codem{else}(plus\ y\ (fois(\codem{pred}x)\ y))\\
		\codem{in}\ &aux\\
		&\codem{in\ }fois
	\end{align*}


	\paragraph{Question 3.}
	En Mini-ML, la fonction factorielle peut s'écrire ainsi
	let rec f = $\lambda x^{int}$.\\
	if x=0 then 1 else fois x (f (\code{pred}x))\\
	in f\\

	\paragraph{Question 4.}
	Même s'il n'est pas possible d'écrire\\
	let (x,y) =(2,3) in x\\
	en Mini-ML, on peut s'en sortir en utilisant les fonctions \code{fst} et \code{snd} de la façon suivante\\
	let c = (2,3) in let x = fst c in let y = snd c in x

	\paragraph{Question 5.}
	Comme en OCaml le système de type est correct par rapport à la sémantique à petit pas de Mini-ML, c'est-à-dire qu'il n'y a pas de problème de type lors de l'execution d'un programme bien typé. Ainsi le système de type nous assure qu'il n'est pas nécessaire de vérifier les types à l'éxecution.

	\paragraph{Question 6.}

	\paragraph{Question 7.}

	\section{Implémentation de Mini-ML}
	\paragraph{Question 11.}





\end{document}
